% Pour flécher les équations
\newif\ifclipme\clipmetrue
\tikzset{labelstyle/.style={LabelStyle/.append style={#1}},linestyle/.style={LineStyle/.append style={#1}}}
\tikzset{LabelStyle/.initial={},LineStyle/.initial={}}

\newcommand{\mathWithDescription}[4][]{{%
    \tikzset{#1}%
    \tikz[baseline, node distance=0.5cm]{
        \node[anchor=base] (m#4) {$\displaystyle#2$};
        \ifclipme\begin{pgfinterruptboundingbox}\fi
            \node[below=of m#4, font=\strut, red, LabelStyle] (l#4) {#3};
            \draw[->, red, thick, >=stealth, LineStyle] (l#4) to (m#4);
        \ifclipme\end{pgfinterruptboundingbox}\fi
    }%
}}
\newcommand{\mathWithDescriptionStarred}[4][]{{%
    \clipmefalse%
    \mathWithDescription[#1]{#2}{#3}{\theeqnnode}%
}}

\newcounter{eqnnode}
\newcommand{\mathlabel}[3][]{%
   \stepcounter{eqnnode}%
   \mathWithDescription[#1]{#2}{#3}{\theeqnnode}%
   \vphantom{\mathWithDescriptionStarred[#1]{#2}{#3}{\theeqnnode}}%
}

% Différentielle
\newcommand{\dd}{\mathrm{d}}

% Analyse vectorielle
\newcommand{\rot}{\overrightarrow{\mathrm{rot}}\,}		% rotationnel
\newcommand{\ddiv}{\mathrm{div}\,}						% divergence
\newcommand{\grad}{\overrightarrow{\mathrm{grad}}\,}	% gradient