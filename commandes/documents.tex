% Propriétés de la légende des subfloat
\captionsetup[subfloat]{ % package caption + subfloat
  margin          = 0pt,
  position        = b,
  justification   = justified,
  singlelinecheck = true,
  labelfont       = {color=black},
  font            = {color=black}
}



%------------------------------
% Documents
%------------------------------

% Définit un environnement pour réaliser des documents

% environnement flottant docfloat
\DeclareFloatingEnvironment[ % package newfloat
  fileext   = frm,
  placement = {!ht},
  name      = Doc.%,
%  within    = chapter
]{docfloat}
\renewcommand{\thedocfloat}{\arabic{docfloat}}

% propriétés de la légende
\captionsetup[docfloat]{ % package caption + subfloat
  justification   = raggedright,
  margin          = 10pt,
  singlelinecheck = false,
  labelfont       = {color=black!60, bf, sf},
  font            = {color=black!40, bf, sf}
}

% mise en forme du document
\newcommand{\docbkg}{black!15} % couleur de fond des documents
\newcommand\doc[4][\linewidth]{
  \begin{minipage}[t]{#1}
    \caption{#2}
    \label{#3}
    \vspace{-0.5cm}
    \begin{mdframed}[roundcorner=10pt, linecolor=\docbkg, backgroundcolor=\docbkg]
      #4
    \end{mdframed}
  \end{minipage}
}



%------------------------------
% Documents annexes
%------------------------------

% Définit un environnement pour réaliser des documents annexes

% environnement flottant annexefloat
\DeclareFloatingEnvironment[ % package newfloat
  fileext   = frm,
  placement = {!ht},
  name      = Annexe%,
%  within    = chapter
]{annexefloat}
\renewcommand{\theannexefloat}{\Alph{annexefloat}}

% propriétés de la légende
\captionsetup[annexefloat]{ % package caption + subfloat
  justification   = raggedright,
  margin          = 10pt,
  singlelinecheck = false,
  labelfont       = {color=black, bf, sf},
  font            = {color=black, sf}
}

% mise en forme du document annexe
\newcommand\annexe[4][\linewidth]{
  \begin{minipage}[t]{#1}
    \caption{#2}
    \label{#3}
    \vspace{-0.5cm}
    \begin{mdframed}[roundcorner=10pt, linecolor=black, linewidth=1mm, backgroundcolor=white]
      #4
    \end{mdframed}
  \end{minipage}
}



%------------------------------
% Points de méthode
%------------------------------

% Définit un environnement pour réaliser des points de méthode

% environnement flottat methodoenv
\DeclareFloatingEnvironment[ % package newfloat
  fileext   = frm,
  placement = {H},
  name      = Fiche méthode:%,
]{methodoenv}

% propriétés de la légende
\captionsetup[methodoenv]{ % package caption + subfloat
  justification   = raggedright,
  margin          = 10pt,
  singlelinecheck = false,
  labelformat     = empty,
  labelfont       = {color=darkgreen!80, bf, sf},
  font            = {color=darkgreen!80, bf, sf}
}

% mise en forme du point de méthode
\newcommand\methodo[2]{
	\begin{methodoenv}
	  \begin{minipage}[t]{\linewidth}
	    \caption{\faWrench~Méthode~:~#1}
	    \vspace{-0.5cm}
	    \begin{mdframed}[roundcorner=10pt, linecolor=darkgreen!20, backgroundcolor=darkgreen!20]
	      #2
	    \end{mdframed}
	  \end{minipage}
	\end{methodoenv}
}




%------------------------------
% Données
%------------------------------

% Définit un environnement pour indiquer des données

% environnement flottant dataenv
\DeclareFloatingEnvironment[ % package newfloat
  fileext   = frm,
  placement = {!h},
  name      = Données:%,
]{dataenv}

% propriétés de la légende
\captionsetup[dataenv]{ % package caption + subfloat
  justification   = raggedright,
  margin          = 10pt,
  singlelinecheck = false,
  labelformat     = empty,
  labelfont       = {color=lightorange, bf, sf},
  font            = {color=lightorange, bf, sf}
}

% mise en forme des données
\newcommand\data[2]{
	\begin{dataenv}
	  \begin{minipage}[t]{\linewidth}
	    \caption{\faBook~Données~:~#1}
	    \vspace{-0.5cm}
	    \begin{mdframed}[roundcorner=10pt, linecolor=lightorange!50, backgroundcolor=lightorange!50]
	      #2
	    \end{mdframed}
	  \end{minipage}
	\end{dataenv}
}



%------------------------------
% Points maths
%------------------------------

% Définit un environnement pour réaliser des points mathématiques

% environnement flottant mathenv
\DeclareFloatingEnvironment[ % package newfloat
  fileext   = frm,
  placement = {!ht},
  name      = Point maths:%,
]{mathenv}

% propriétés de la légende
\captionsetup[mathenv]{ % package caption + subfloat
  justification   = raggedright,
  margin          = 10pt,
  singlelinecheck = false,
  labelformat     = empty,
  labelfont       = {color=darkgreen!80, bf, sf},
  font            = {color=darkgreen!80, bf, sf}
}

% mise en page du point maths
\newcommand\pointmath[3][\linewidth]{
	\begin{mathenv}
	  \begin{minipage}[t]{#1}
	    \caption{\faBarChart~Point maths~:~#2}
	    \vspace{-0.5cm}
	    \begin{mdframed}[roundcorner=10pt, linecolor=darkgreen!20, backgroundcolor=darkgreen!20]
	      #3
	    \end{mdframed}
	  \end{minipage}
	\end{mathenv}
}



%------------------------------
% Définition
%------------------------------

% mise en page des définitions
\mdfdefinestyle{definition}{
	linecolor                 = darkred,
	linewidth                 = 1pt,
	frametitle                = Définition,
	frametitlefont            = {\normalfont\bf\sffamily\color{white}},
	frametitlebackgroundcolor = darkred,
	backgroundcolor           = white
}



%------------------------------
% Lois et principes
%------------------------------

% mise en page des lois et principes
\mdfdefinestyle{principe}{
	linecolor                 = DarkBlueChapter,
	linewidth                 = 1pt,
	frametitle                = Principe,
	frametitlefont            = {\normalfont\bf\sffamily\color{white}},
	frametitlebackgroundcolor = DarkBlueChapter,
	backgroundcolor           = white
}